\documentclass[11pt]{article}
\usepackage{latexsym,amsmath,amssymb,amsfonts,longtable,graphicx}
\pagestyle{empty}
\setlength{\oddsidemargin}{0cm}
\setlength{\evensidemargin}{0cm}
\setlength{\topmargin}{-2.0cm}
\setlength{\textwidth}{16cm}
\setlength{\textheight}{24.5cm}
\setlength{\parindent}{0cm}
\setlength{\parskip}{0.1cm}

\begin{document}

An optimal plan for air cargo p1 is:\\

Load(C1, P1, SFO)\\
Load(C2, P2, JFK)\\
Fly(P1, SFO, JFK)\\
Fly(P2, JFK, SFO)\\
Unload(C1, P1, JFK)\\
Unload(C2, P2, SFO)\\

An optimal plan for air cargo p2 is:\\

Load(C1, P1, SFO)\\
Load(C2, P2, JFK)\\
Load(C3, P3, ATL)\\
Fly(P1, SFO, JFK)\\
Fly(P2, JFK, SFO)\\
Fly(P3, ATL, SFO)\\
Unload(C3, P3, SFO)\\
Unload(C2, P2, SFO)\\
Unload(C1, P1, JFK)\\

An optimal plan for air cargo p3 is:\\

Load(C1, P1, SFO)\\
Load(C2, P2, JFK)\\
Fly(P1, SFO, ATL)\\
Load(C3, P1, ATL)\\
Fly(P2, JFK, ORD)\\
Load(C4, P2, ORD)\\
Fly(P2, ORD, SFO)\\
Fly(P1, ATL, JFK)\\
Unload(C1, P1, JFK)\\
Unload(C2, P2, SFO)\\
Unload(C3, P1, JFK)\\
Unload(C4, P2, SFO)\\



The performance metrics for three breadth first, depth first and uniform cost search are shown in Table~\ref{uninformed} below.
Overall, depth first search performs better in terms of the time spent, number of goal tests and number of node expansions, however, since 
it searches by first executing actions to get to states deeper in the tree, it fails to provide an optimal solution. Uniform cost search and breadth first search 
both provide optimal solutions, and their performances are similar in terms of number of goal tests. Uniform cost search peforms worse than breadth first search in terms of node expansions.
However, as the problems gets more challenging, uniform cost search performs better than breadth first search in terms of time spent. The non-optimality of the solution provided by depth first search and the optimality of the solution privided by breadth first and uniform cost search aligns with what we lerned in the search comparison module of the search chapter in class. \\

\begin{table}[h!]
\caption{Performance metrics of three uninformed search techniques}
\centering
\resizebox{\textwidth}{!}{
\begin{tabular}{||c | c | c| c| c|c ||} 
 \hline
Problem &Technique &  No of node expansions & No of goal tests & Time elapsed in seconds & Optimal Solution Obtained \\ \hline
air cargo p1 & Breadth first search & 43 & 56 & 0.0209 & Yes\\ \hline
air cargo p1 & Depth first graph search & 21 & 22 & 0.0096 & No\\ \hline
air cargo p1 & Uniform cost search & 55 & 57 & 0.0249 & Yes\\ \hline
air cargo p2 & Breadth first search & 3346 & 4612 & 9.0769 & Yes\\ \hline
air cargo p2 & Depth first graph search & 107 & 108 & 0.2132 & No\\ \hline
air cargo p2 & Uniform cost search & 4853 & 4855 & 7.5028 & Yes\\ \hline
air cargo p3 & Breadth first search & 14663 & 18098 & 67.5347 & Yes\\ \hline
air cargo p3 & Depth first graph search & 408 & 409 & 1.1598 & No\\ \hline
air cargo p3 & Uniform cost search & 17882 & 17884 & 32.0927 & Yes\\ \hline
\end{tabular}}
\label{uninformed}
\end{table}

The performance metrics for A* search using the three different heuristic functions are shown in Table~\ref{heuristic} below. 
We focus on the heuristics h ignore preconditions and h pg level sum. Both heuristics give an optimal solution, but as the problem gets more challenging,
 h ignore preconditions spends less time that h pg level sum. However, h ignore preconditions performs more node expansions and more goal tests. From the search chapter in class we learned that using an admissible heuristic for A* search always lead to an optimal solution. Sections 10.2.3 and 10.3.1 of the third edition of the AIMA textbook discuss the admissibility of h ignore preconditions and h pg level sum respectively. Therefore as expected we do obtain optimal soultions using them, \\

\begin{table}[h!]
\caption{Performance metrics of the three heuristics using informed search technique A*}
\centering
\resizebox{\textwidth}{!}{
\begin{tabular}{||c | c | c| c| c|c ||} 
 \hline
Problem &Heuristic &  No of node expansions & No of goal tests & Time elapsed in seconds & Optimal Solution Obtained \\ \hline
air cargo p1 & h1 & 55 & 57 & 0.0251 & Yes\\ \hline
air cargo p1 & h ignore preconditions & 41 & 43 & 0.0268 & Yes\\ \hline
air cargo p1 & h pg levelsum & 39 & 41 & 0.9640 & Yes\\ \hline
air cargo p2 & h1 & 4853 & 4855 & 7.3531 & Yes\\ \hline
air cargo p2 & h ignore preconditions & 1450 & 1452 & 2.7441 & Yes\\ \hline
air cargo p2 & h pg levelsum & 1129 & 1131 & 301.4720 & Yes\\ \hline
air cargo p3 & h1 & 17882 & 17884 & 31.3946 & Yes\\ \hline
air cargo p3 & h ignore preconditions & 5034 & 5036 & 10.6019 & Yes\\ \hline
air cargo p3 & h pg levelsum & 2025 & 2027 & 1038.5874 & Yes\\ \hline
\end{tabular}}
\label{heuristic}
\end{table}

To compare performance of uninformed search to heuristic search techniques, its perhaps better to focus on the more challenging air cargo p3. 
In this regard, depth first search is the best technique in terms of time spent, number of node expansions and number of goal tests,
 however it fails to provide an optimal solution. Among the techniques that give an optimal solution the h ignore precondition heuristic performed
 best in terms of time spent, however h pg level sum performs best in terms of number of node expansions and number of goal tests. Overall the best heuristic 
 is h ignore preconditions.
\end{document}
\grid
